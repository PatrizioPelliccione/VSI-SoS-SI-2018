\section{Engineer an SoS: challenges and opportunities}

%Connected vehicles will benefit from Intelligent Transport Systems (ITS), Smart Cities, and Internet of Things (IoT). 



Connected vehicles  will combine data from inside vehicle
with external data coming from the environment (other vehicles, the road, signs,
and the cloud).
Future transportation systems will be a heterogeneous mix of items with varying connectivity and interoperability. A mix of new technologies and legacy systems will co-exist to realize a variety of scenarios involving not only connected vehicles but also smart roads, pedestrians, cyclists, and so on. In other words, future transportation systems are expected to be system of systems, where each constituent system can act as a standalone system, but where the cooperation among the constituent systems will enable new emerging and promising scenarios.  
%smart
%traffic control, better platooning coordination, and enhanced safety in general.
In today's vehicles, the driver plays a fundamental role. Besides controlling
the vehicle, she/he is also expected to recover the vehicle from failures. With fully
autonomy (SAE level 5), the driver is no longer in the loop. The vehicle, besides
controlling the basic driving maneuvers, should handle all possible situations during
driving. As the driver is not monitoring the vehicle and is not a fallback option
anymore as in the lower levels of automation, it is expected that nothing goes wrong.
For the higher levels of automation (i.e., levels 3 - 5), additional players enter the
ecosystem: companies that focus on communication aspects will play a big role in
this ecosystem, as communication will be an important basis for highly automated
driving~\cite{Knauss2017challenges}. Communication allows the vehicles to ensure that they have redundant
information sources (i.e., several sources of information) to ensure safety, reliability,
and quality of autonomous vehicles. In other words, autonomous cars are becoming part of an autonomous system of systems. An autonomous car needs to constantly communicate with other cars and the infrastructure, via sensors and the cloud. As foreseen by IBM, unprecedented complexity will lead to unprecedented cooperation\footnote{ \url{https://www.ibm.com/blogs/internet-of-things/iot-design-for-system-of-systems}}. This is also confirmed by some discussions we had with several Swedish companies (between others Volvo Car Group, Volvo Trucks, Ericsson,
Axis communications, Jeppesen - Boing); all of them claimed that one of the business challenge for industry
in the near future is to build and provide new functionalities and/or services in a short time scale by easily integrating
the system with other systems and to continuously maintain and evolve the system under guarantees of
quality of service.


%As can be seen in this example, data plays an
%essential role in autonomous vehicles. Hence, companies that focus on data will be
%additional players in the ecosystem~\cite{Knauss2017challenges}. 







As anticipated in Section~\ref{sec:sota}, SoSE is an emerging discipline of the last years that addresses the development,
operation and maintenance of SoSs. Even though the community around SoSE is active, SoSE is a
young discipline, and general principles and theoretical foundations remain to be discovered. 
Engineering SoSs requires cooperation among companies, and there is the need of standards. This is testified by the investigation of Toyota about intelligent transport systems\footnote{ \url{www.toyota-global.com/innovation/smart_mobility_society/intelligent_transport_systems}} and by the 
pn-European project SOCRATES2.0 (System of Coordinated Roadside and Automotive Services for Traffic Efficiency and Safety)\footnote{\url{https://socrates2.org/}}, which aims at setting new standards to share and integrate traffic information to enable effective traffic management and navigation services.

Once defined a collaboration among the players of the system of systems, many questions need to be answered,. e.g.:

\begin{itemize}
\item Should the SoS be Virtual, Collaborative, Acknowledged, or Directed, or a mix of them?
\item What is the extent of ``controllability" of the constituent systems?
\item Should the collaboration be established for a limited time in response to or to solve specific situations, like hazards in connected safety scenarios?
\item Should the collaboration be extemporary and spontaneous?
\item Who is engineering the SoS, one specific OEM, a group of OEMs, road authorities, etc.?
\item Who is the owner of the SoS?
\item Who is the manager of the SoS?
\end{itemize}

Moreover, some of the scenarios that are supported by the SoS might be safety-critical. Consequently, SoS engineering needs to be grounded on a rigorous conceptual framework, so to enable: (i) precise and unambiguous definition of SoS properties; (ii) runtime
verification so to control, avoid, or block negative/harmful unanticipated emergent behaviours (emergent behaviours will be explained later in this section), and (iii) the development
of tools for managing the SoS. 
In order to understand the complexity of the problem, we mention two challenging characteristics of SoSs: 

\begin{itemize}
\item Uncertainty about constituents - SoSs need to manage unavoidable imprecise
and uncertain information about constituent systems; 
\item Independence of constituents - constituent systems
are independent entities for what concern operation, management, and evolution. 
\end{itemize}

Each constituent has its own
objectives, possibly conflicting with objectives of the other constituents or of the SoS. An SoS has a potentially
limited authority that it exercises on the constituent systems to drive them towards its objectives. 

An open problem of SoSs is how to provide justification of the reliance on emergent properties~\cite{Fitzgerald2014}. Emergent
properties are the result of synergistic collaboration between constituent systems, and are those properties
that cannot be expressed at the level of a single constituent system, but require observations of phenomena at the
SoS boundary~\cite{Fitzgerald2014}. There are two different types of emergence: weak emergence [6], when an emergent property
can be reduced to its constituents, and strong emergence, when an emergent property cannot be traced to any
direct cause (a classic example is the consciousness emerging as a property of the brain~\cite{Fitzgerald2014}). Like~\cite{EmergenceAndRefinement,Fitzgerald2014}  we
confine our attention to weak emergence, and hereafter, we use the term emergence to refer to weak emergence.
Emergence can be either anticipated, meaning that it is defined at design-time, or unanticipated, meaning
that it is not purposely or consciously designed-in or surprising to the developers and users of the SoS. Consequences
of the unanticipated emergent behavior may be viewed as negative/harmful, positive/beneficial, or
neutral/unimportant by stakeholders of the SoS~\cite{DDUSA2012}. Existing approaches to design and verify dependable systems\footnote{Dependability is a generic concept including attributes like reliability, availability, safety, integrity, maintainability, etc.~\cite{dependability}.} 
work pretty well with closed and unchanging systems. These techniques become inadequate for assessing
and justifying SoS reliance on emergent properties since SoSs are composed of autonomous constituents
whose behaviour can neither be predicted nor controlled~\cite{Strigini12}. Investigation is needed to provide ways for characterizing, quantifying and measuring a system
behavior in the presence of perturbations~\cite{Strigini12}.

One of the most frequently used techniques to explore emergence is simulation~\cite{DeWolf2005}. The work in~\cite{Zambonelli04} advocates that different engineering techniques are required at different scales,
including software engineering and formal methods. Self-adaptive systems~\cite{de2013software,Author:2009tt,Salehie2009} are systems that autonomously
decide how to adapt at runtime to meet environment (context) and user changes and threats. Uncertainty about
constituents and independence of constituents make most of the techniques proposed for self-adaptive system
inapplicable for SoSs. Providing guarantees that an SoS will achieve a global goal in predictable and dependable
ways, through the collaboration of autonomous and independent constituent systems with different (and
potentially conflicting) local goals remains an open problem~\cite{Delgado2004,Leucker2009293,ZHANG20185}. There is the need of evidence on which
to base reliance on emergent properties and on SoS-level behaviours.

 










