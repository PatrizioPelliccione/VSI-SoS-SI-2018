\section{Challenges of the Automotive Domain}\label{sec:challenges}

Many emergent business and technological needs are today shocking the automotive domain. Traditional OEMs are realizing that they should increasingly invest on software since 80\% to 90\% of the innovation within the automotive industry
is driven by electronics and software~\cite{softwareAutomotive}. The development of modern cars has to cope with a large amount of concerns, including safety, security, variability management, electrification, autonomy, networking, costs, weight, etc. Moreover, an increasing amount of people is involved in software development projects and this imposes additional challenges since the development and design can no longer be controlled, or even understood, in detail by
a single group. Here in the following we discuss some of the main challenges of the automotive domain.

\noindent {\bf Autonomy} -
Intelligent vehicles will reach full automation, freeing the driver from performing
any task. This is a path that we will only be reached gradually. There are six levels
of vehicle automation. At level 0 a human driver is responsible for steering, throttle, and breaking and this level is referred as no automation. 
At level 1, known as driver assistance, the vehicle can perform some control function but not everywhere. At level 2, partial automation, the vehicle can handle steering, throttle, and breaking but the driver
is expected to monitor the system and take over in case of faults. In these first three levels, the human driver
monitors the environment. In the next three levels, the monitoring of the environment is under the control of the driving system. At level 3, conditional automation, the vehicle monitors the surroundings and notifies the driver if manual control is needed. At level 4, known as high automation, the vehicle is fully autonomous but only in defined use cases. Finally, the full automation is reached at level 5, when the driver has only to set the destination and them the vehicle will handles any surrounding and make any kind of decision on the way.

To date, almost all of the
vehicles in circulation settle on the levels comprised between level 0 and 2 (level
2 is defined as ``partial automation"), in which the systems are limited to assist the
driver without replacing him.
The path towards fully autonomous vehicles
is marked by incremental advances in different categories of assistance. Reaching a
level 5 of autonomy is more an evolutionary path rather than a revolution from one
day to another.
OEMs are establishing
partnerships with suppliers focusing on autonomous vehicles. For example,
Zenuity is a joint venture on self-driving cars between Volvo Cars and the supplier
Autoliv. BMW announced the alliance with Intel and MobileEye with the plan to
bring self-driving cars on the road by 2021. Mercedes joins the forces with Bosch,
one of the biggest automotive supplier, to develop level 4 and level 5 vehicles.
Google and Tesla are among the pioneers to push the autonomous driving related
software into the market but at least 33 companies are working on autonomous
vehicles4. Most of them had promised to put the autonomous car on the road by
2021, assuming that the necessary regulations will be put in place by then.


\noindent {\bf Safety} - Safety is one of the major concerns within the automotive domain globally. It has been shown by WHO that road traffic injuries is the top cause for death among people 15-29 years of age~\cite{WHO2015}. When looking at vehicles, the number of sophisticated safety systems is increasing and this has demonstrated a clear safety benefit. But, it is also known that almost half of all fatalities might affect the so called Vulnerable Road Users (VRU), e.g. pedestrians, cyclists, and powered two wheelers. Traditionally safety is focusing on vehicles, and consequently,  VRU do not get the same benefit of technology since they generally are not equipped with safety restraints and active safety systems. VRU are becoming part of future transportation systems through sensors in infrastructure systems but also through connected devices (Internet-of-things).

\noindent{\bf Dependability} - A car being able to act as a member of a SoS will enable many different kinds of services directed to (i) cars, (ii) their drivers and (iii) other end users of future transportation systems, e.g. the VRU. Many of these services must be dependable to ensure safety. It is important to distinguish between the different levels of Quality of Service (QoS) needs and especially safety requirements that these services give rise to. In general, safety-critical services put very high demands on security, trust, and dependability properties in general. Much of the fulfilment of these properties in its turn depends on sufficient QoS of the connectivity between the SoS constituents involved.
 
\noindent {\bf Comfort} - Automation of driving became a big buzz the last years; while safety is given as an argument for automation, the main drivers for it are comfort and efficiency. As an example, 30\% of the traffic in congested areas might be just looking for parking stands \cite{Shoup2011thc}, and this is considered a burden by many drivers. Moreover, the view of the car as a status symbol is changing and, in some parts of the world, young people do no longer consider having a car to be an important value. This trend is attested by many new actors on the market that provide mobility as a service.
 
\noindent {\bf Efficiency} - Traffic jams and congestion have been a growing problem in the world, and the road infrastructures are not able to keep up with the rapid increase of demands in major parts of the world due to various reasons such as space, environment, and cost. Electrification of vehicle power-trains is an increasing trend~\cite{KPMG}, which will lead to a change on how vehicle users need to power their vehicles, hence they need to be linked to a system to avoid long waiting times at charging stations.
 
\noindent {\bf Human Interaction} - When vehicles are automated and become constituent systems of a System of Systems, one challenge is to make them socially acceptable~\cite{vinkhuyzen2016}. How to interact with pedestrians and other VRU is subject to a social code and can even be different within countries. The same goes for negotiations between vehicles, for example merging (zipper) in one lane before a roadwork is subject to cultural aspects. A System of Systems needs also to adapt to evolving needs of future transportation system users in order to be socially acceptable.