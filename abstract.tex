\begin{abstract}
In the last years cars are evolved from a disconnected and ``blind" systems to systems able to sense the surrounding environment, to systems that are connected with other vehicles, the city, pedestrians, cyclists, etc. 
The challenge nowadays is to properly exploit the new opportunities coming from this new dimension without sacrificing safety, security, privacy, etc. Future transportation systems
can be seen as a System of Systems (SoS), where each constituent
system - one of the units that compose an SoS - can act as a standalone
system, but the cooperation among the constituent systems
enables new emerging and promising scenarios.

In this paper we 
investigate how to architect cars so that they can be constituents of
future transportation systems. This work is realized in the context
of two Swedish projects coordinated by Volvo Cars and involving
some universities and research centers in Sweden and many suppliers
of the OEM, including Autoliv, Arccore, Combitech, Cybercom, Knowit, Prevas, \AA F-Technology, Semcom, and Qamcom.
\end{abstract}