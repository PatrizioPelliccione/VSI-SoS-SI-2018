\begin{abstract}
Over the last years cars have evolved from disconnected and ``blind" systems to systems able to sense the surrounding environment and to systems that are connected with other vehicles, the city, pedestrians, cyclists, etc. 
The current challenge is to properly exploit these new opportunities without sacrificing key qualities such as safety, security, and privacy.
Future transportation systems can be seen as a System of Systems (SoS), where each constituent
system -- one of the units that compose an SoS -- can act as a standalone
system, but the cooperation among the constituent systems
enables new emerging and promising scenarios.

In this paper we 
investigate how to architect cars so that they can be constituents of
future transportation systems. 
Then, we point out the opportunity and need for a collaboration among different OEMs and with other relevant stakeholders, like road authorities and smart cities, to properly engineer systems of systems composed of cars, roads, pedestrians, etc.
This work is realized in the context
of two Swedish projects coordinated by Volvo Cars and involving
some universities and research centers in Sweden and many suppliers
of the OEM, including Autoliv, Arccore, Combitech, Cybercom, Knowit, Prevas, \AA F-Technology, Semcom, and Qamcom.
\end{abstract}